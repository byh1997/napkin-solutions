\noindent\textbf{1A.}  A finite group clearly does not have a subgroup isomorphic to itself, because the two groups have different orders.  Therefore, only infinite groups may have subgroups isomorphic to themselves.  \qed

\noindent\textbf{1B.}  Let $g\in G$ be any element of the group, and let $g_1,g_2,\dots,g_{|G|}$ be an enumeration of the elements of $G$.  Recall that left-multiplication by $g$ permutes the elements of $G$.  Therefore, 
\[g_1g_2\cdots g_{|G|}=(gg_1)(gg_2)\cdots (gg_{|G|}).\]
Since $G$ is abelian, we can rearrange the right-hand side and cancel $g_1g_2\cdots g_{|G|}$, leaving $g^{|G|}=1$.  \qed

\noindent\textbf{1C.}  Observe that $D_6$ and $S_3$ both represent the symmetries of a triangle, and are therefore isomorphic. 

The group $D_{24}$, which represents the symmetries of a regular dodecagon, has an element of order 12.  Since $S_4$ does not have an element of order 12, the groups cannot be isomorphic. \qed 

\noindent\textbf{1D.}  By Lagrange's Theorem, all elements have order dividing $p$; thus, all elements have order $1$ or $p$.  Since exactly one element, the identity, has order 1, there is at least one element $g\in G$ with order $p$.  Then, the elements $1,g,g^2,\dots,g^{p-1}$ are distinct, so $G$ is isomorphic to $\Z_p$.  \qed

\noindent\textbf{1E.}  a) Let $G=\{g_1, \dots, g_{|G|}\}$.  By the cancellation law, left-multiplication by any element $g_i\in G$ permutes the set $\{g_1,\dots,g_{|G|}\}$, and we may identify $g_i$ with the corresponding permutation of $\{1,\dots,|G|\}$.  Left-multiplying the set $\{g_1,\dots,g_{|G|}\}$ by $g_i,g_j\in G$ is the same as left-multiplying by $g_j$, then by $g_i$, so multiplication in $G$ corresponds to composition in $S_{|G|}$.  This is an isomorphism.  

b) By (a), $G$ is isomorphic to some subgroup of $S_{|G|}$.  So, $G$ is also isomorphic to the group of corresponding permutation matrices in $GL_{|G|}(\R)$.  

\noindent\textbf{1F.}  Recall that $S_3$ is defined as the group generated by $x,y$ under the relations $x^3=y^2=1$, $yx=x^2y$.  

We identify each white chip with $x$ and each black chip with $y$, so the row of chips, read left to right, is a product of $x$'s and $y$'s.  Observe that 
\begin{align*}
xxx&=yy\\
xxy&=yx\\
yxx&=xy\\
yxy&=xx.
\end{align*}
Hence, the product of the row of chips is invariant under the given operation.  

If $n\equiv 1\pmod 3$, the product of the chips is $x^n=x$.  So, if it is possible to reach a state with two chips, the product of those chips must be $x$.  

Observe that the number of black chips is always even.  Thus, if two chips remain in the end, they must be both black or both white.  

But, $xx=x^2\neq x$ and $yy=1\neq x$.  This is a contradiction.  \qed